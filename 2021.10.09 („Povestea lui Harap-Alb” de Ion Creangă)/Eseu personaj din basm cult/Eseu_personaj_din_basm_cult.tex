% Document preamble
\documentclass[
12pt,
a4paper
]{article}

% Packages
\usepackage[T1]{fontenc}            % Proper hyphenation
\usepackage[utf8]{inputenc}         % UTF-8 support
\usepackage[romanian]{babel}        % Romanian characters support
\usepackage{indentfirst}            % Add paragraph indentation even after a section
\usepackage[margin=2.7cm, marginparwidth=2cm, marginparsep=3mm]{geometry} % Make document margins smaller
\usepackage{marginnote}             % Notes on the margins of a document (more advanced \marginpar)
\usepackage{titlesec}               % Customize titles

% Custom titles, sections, subsections etc. format
\titleformat*{\section}{\large\bfseries}
\titleformat{\subsection}{\normalfont\normalfont\bfseries}{}{0pt}{}

% Page numbering
%\pagenumbering{gobble} % uncomment if you want to disable it

% Custom commands
% Format: \newcommand{\command}{action (add '\ ' or '{}' if it won't add a space properly)}
\newcommand{\operatitle}{\textbf{\textit{„Povestea lui Harap-Alb”\ }}} % title of the commented opera
\newcommand{\operaauthor}{Ion Creangă\ }                               % author of the commented opera

% Basic document info
\title{Eseu despre particularitățile de construcție a personajului principal dintr-un basm cult studiat}
\date{}   % Show no date in the title
\author{} % Empty author to not get a warn about missing author

\begin{document}
\maketitle % Show the title
%\reversemarginpar % put margin notes on left instead of on right


% Beginning of text

\subsection{Context}

\operatitle de \operaauthor este un basm cult, publicat în revista „Convorbiri literare”, în anul 1877.

{\footnotesize(Introducerea este comună cu aceea a comentariului!)}

\section{Precizarea statutului social, psihologic, moral etc. al personajului ales}

\marginnote{erou atipic}[0.8cm]
Harap-Alb este protagonistul basmului, întruchipare a binelui, dar este un erou atipic de basm, deoarece este lipsit de însușiri supranaturale, fiind construit realist, ca o ființă complexă, care învață din greșeli și progresează. De aceea este personaj \textit{„rotund”}, ieșind din stereotipia superiorității mezinului. Este personaj \textit{„tridimensional”}, căci iese din tipar, surprinde, ca, de exemplu, atunci când îi dă calului cu frâul în cap sau râde împreună cu ceilalți de Gerilă, în casa de aramă.

\marginnote{de la naivitate la în\-țe\-lep\-ciu\-ne}[0.8cm]
Statutul inițial al eroului este cel de neinițiat. Mezinul craiului este naiv, nu știe să distingă adevărul de minciună, să vadă caracterul unui om dincolo de aparențe. Are nevoie de experiența vieții spre a dobândi înțelepciune. Se deosebește de frații săi, încă de la început, prin bunătate, calitate răsplătită de sfaturile Sfintei Duminici, după ce o miluiește cu un ban. Deși are calitățile necesare unui viitor împărat, în viziunea autorului, fiind \textit{„cel mai \textbf{vrednic} dintre nepoți”}, cum spune Împăratul Verde, acestea nu sunt individualizate de la început, ci și le descoperă prin intermediul probelor la care este supus, când dovedește generozitate, prietenie, responsabilitate.

Numele Harap-Alb semnifică sclav-alb, rob de origine nobilă, dar și condiția de învățăcel, faptul de a fi supus inițierii, transformării. Cele trei nume ale lui corespund, în plan compozițional, celor trei etape ale drumului inițiatic: la început - \textit{„fiul craiului”, mezinul (naivul)}; pe parcursul călătoriei - \textit{Harap-Alb (învățăcelul)}; la sfârșit - \textit{împăratul (inițiatul)}.

\section{Ilustrarea a patru elemente de structură și de compoziție ale basmului, semnificative pentru realizarea personajului din basmul cult studiat {\footnotesize(de exemplu: acțiune, conflict, relații temporale și spațiale, incipit, final, tehnici narative, perspectivă narativă, registre stilistice, limbajul personajelor etc.)}}

Titlul sugerează tema basmului: maturizarea mezinului craiului. Concret, eroul parcurge o aventură imaginară, un drum al maturizării, în care dobândește valori morale și etice, pentru ca la final să devină împărat (basmul are, așadar, valoare de Bildungsroman).

Acțiunea se desfășoară linear, prin înlănțuirea secvențelor narative, respectă modelul structural al basmului și implică prezența fabulosului, dar mai puțin în ceea ce-l privește strict pe mezin. Conflictul dintre bine și rău se încheie prin victoria forțelor binelui. Este amplificat procedeul compozițional al triplicării în cazul probelor pe care eroul le are de trecut. Sunt prezente cifre și obiecte magice.

Personajele (oameni, dar și \textit{„ființe himerice”} cu comportament omenesc) îndeplinesc, prin raportare la erou, o serie de funcții (antagonist, ajutoare, donatori), ca în basmul popular, dar sunt individualizate, mai ales, prin limbaj.

\section{Ilustrarea trăsăturilor personajului ales, prin secvențe \\ narative/situații semnificative sau prin citate comentate}

Protagonistul este construit prin procedee de caracterizare directă (de către narator, de către alte personaje și prin autocaracterizare) și de caracterizare indirectă, prin fapte, limbaj, gânduri, relații cu alte personaje, nume.

\marginnote{ajutoare și donatori}[0.8cm]
Eroul este sprijinit de ajutoare și donatori: ființe cu însușiri supranaturale (Sfânta Duminică), animale fabuloase (calul năzdrăvan, crăiasa furnicilor și a albinelor), făpturi himerice (cei cinci tovarăși) sau obiecte miraculoase (aripile crăieselor, smicelele de măr, apa vie, apa moartă) și se confruntă cu răufăcătorul/personajul antagonist (Spânul), care are și funcție de trimițător. Personajul căutat este fata de împărat.

Cu excepția eroului care este văzut în evoluție, de la naivitate la înțelepciune, celelalte personaje au o trăsătură dominantă.

Primele întâlniri cu inițiatorii săi, Sfânta Duminică, apoi calul năzdrăvan și Spânul, pun în lumină naivitatea, incapacitatea de a distinge adevărul de aparențe.

După ce iese din împărăția tatălui său, crăișorul se rătăcește în pădurea-labirint. Încalcă sfatul dat de tată (interdicția să se ferească de omul spân și de omul roș) și își ia drept călăuză un spân viclean. În episodul coborârii în fântână, naratorul surprinde lipsa de experiență a tânărului, prin caracterizare directă. Naivitatea tânărului face posibilă supunerea prin vicleșug.

\marginnote{întâlnirea cu Spânul}[0.8cm]
Antagonistul (răufăcătorul) îl închide pe tânăr în fântână și îi cere, pentru a-l lăsa în viață, să facă schimb de identitate, să devină robul lui și să jure \textit{„pe ascuțișul paloșului”} (sugestie a unui cod al conduitei cavalerești) să-i dea ascultare întru toate, \textit{„până când va muri și iar va învia”}, condiționare paradoxală, dar care arată și calea de eliberare. De asemenea, Spânul îi dă fiului de crai numele de Harap-Alb.

Spânul personifică răul, dar este și inițiatorul pretențios: cu cât încercările la care îl supune pe tânăr sunt mai grele, cu atât eroul dovedește calități morale care conturează portretul viitorului împărat.

Spânul îi cere să aducă \textit{„sălăți”} din Grădina Ursului, pielea cu pietrele prețioase din Pădurea Cerbului și pe fata Împăratului Roș. Harap-Alb își demonstrează curajul și destoinicia în trecerea primelor două probe cu ajutorul obiectelor magice de la Sfânta Duminică.

\marginnote{prietenia}[0.8cm]
Pentru aducerea fetei Împăratului Roș este sprijinit de adjuvanți și donatori. Ca și în cazul milosteniei față de bătrâna cerșetoare, aceste personaje îl ajută pentru că mai întâi el și-a dovedit generozitatea și îndemânarea (față de roiul de albine), bunătatea și curajul (la întâlnirea cu nunta de furnici), prietenia/spiritul de tovărășie (față de Gerilă, Flămânzilă, Setilă, Ochilă și Păsări-Lăți-Lungilă).

Ultima probă presupune mai multe serii de probe, prin care Împăratul Roș tinde să îndepărteze ceata de pețitori (casa încălzită, ospățul, alegerea macului de nisip) și care o vizează direct pe fată (fuga nocturnă a fetei transformată în pasăre, ghicitul fetei/motivul dublului și proba impusă chiar de fată: aducerea unor obiecte magice, \textit{„trei smicele de măr dulce și apă vie și apă moartă de unde se bat munții în capete”}).

\marginnote{onestitatea}[0.3cm]
Pentru erou, aducerea fetei Împăratului Roș la Spân este cea mai dificilă încercare, pentru că pe drum se îndrăgostește de ea, dar, onest, își respectă jurământul făcut și nu-i mărturisește adevărata sa identitate.

La întoarcerea la curtea lui Verde-Împărat are loc recunoașterea și transfigurarea eroului, dar și demascarea și pedepsirea răufăcătorului.

Spânul este demascat de fată, o \textit{„farmazoană”} (are puteri supranaturale). El îi taie capul lui Harap-Alb și îl dezleagă astfel pe erou de jurământul supunerii, semn că inițierea este încheiată, iar calul îl omoară pe răufăcător. Eroul este înviat de fată cu ajutorul obiectelor magice. Învierea este o trecere la o altă identitate: aceea de împărat iubit, slăvit și puternic. Pentru vrednicia lui, primește răsplata cuvenită: nunta și împărăția.

%\section{Susținerea unei opinii despre modul în care se reflectă o idee sau tema basmului cult în construcția personajului}

%În opinia mea, trecerea protagonistului prin încercări dificile, ca și experiența con\-di\-ți\-ei umilitoare de rob la dispoziția unui stăpân nedrept, conturează sensul didactic al basmului, care este exprimat de Sfânta Duminică. Potrivit viziunii despre lume a scriitorului, calitățile necesare unui împărat \textit{„mare și tare”} sunt înțelepciunea, bunătatea și cinstea, calități dovedite de \textit{„omul de soi bun”} pe parcursul unor probe dificile.

\subsection{Concluzie}

În concluzie, deși este un personaj de basm, eroul nu reprezintă doar tipul voinicului, ca Făt-Frumos din basmele populare, ci este și un \textit{„om de soi bun”} (G. Călinescu), eroul \textit{„vrednic”} (cum spune Verde-Împărat) care traversează o serie de probe, se maturizează și devine împărat. Basmul poate fi, astfel, considerat un Bildungsroman.
\end{document}
