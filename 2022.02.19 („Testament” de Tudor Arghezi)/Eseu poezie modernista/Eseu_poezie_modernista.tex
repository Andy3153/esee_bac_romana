%%
%% Basic LaTeX template by Andy3153
%% created   04/10/21 ~ 18:21:32
%% modified1 15/10/21 ~ 23:15:05
%% modified2 03/11/21 ~ 18:37:45
%%

% Document preamble
\documentclass[
12pt,
a4paper
]{article}

% Packages
\usepackage[
margin=2.7cm,                % Margin size
marginparwidth=2cm,          % Margin note size
marginparsep=3mm             % Space between margin and text
]{geometry}
\usepackage[utf8]{inputenc}  % UTF-8 support
\usepackage[T1]{fontenc}     % Proper hyphenation
\usepackage[romanian]{babel} % Romanian characters support
\usepackage{indentfirst}     % Add paragraph indentation even after a section
\usepackage{marginnote}      % Notes on the margins of a document (more advanced \marginpar)
\usepackage{titlesec}        % Customize titles

% Custom titles, sections, subsections etc. format
\titleformat*{\section}{\large\bfseries}
\titleformat{\subsection}{\normalfont\normalfont\bfseries}{}{0pt}{}

% Page numbering
%\pagenumbering{gobble} % uncomment if you want to disable it

% Custom commands
% Format: \newcommand{\command}{action (add '\ ' or '{}' if it won't add a space properly)}
\newcommand{\rom}[1]{\uppercase\expandafter{\romannumeral #1\relax}} % Roman numerals
\newcommand{\operatitle}{\textbf{\textit{„Testament”\ }}} % title of the commented opera
\newcommand{\operaauthor}{Tudor Arghezi\ } % author of the commented opera

% Basic document info
\title{Eseu cu privire la tema și viziunea despre lume într-o poezie modernistă/artă poetică studiată}
\date{}   % Show no date in the title
\author{} % Empty author to not get a warn about missing author

\begin{document}
\maketitle % Show the title
%\reversemarginpar % put margin notes on left instead of on right


% Beginning of text

\subsection{Context}

Poezia \operatitle de \operaauthor face parte din seria artelor poetice moderne ale literaturii române din perioada interbelică. Poezia este așezată în fruntea primului vulum arghezian, \textbf{\textit{„Cuvinte potrivite”}} (1927), și are rol de program literar, realizat însă cu mijloace artistice.

\operaauthor este un înnoitor al limbajului poetic, prin încălcarea convențiilor și a regulilor. Perticulatități ale modernismului în poezia lui sunt: \underline{estetica urâtului}, limbajul șocant, cu neașteptate asocieri lexicale și semantice, fantezia metaforică, înnoirile prozodice. Ține de tradiționalism ideea legăturii dintre generații și opțiunea pentru tematica socială.

\section{Evidențierea trăsăturilor care fac posibilă încadrarea textului poetic studiat într-un curent cultural/literar, într-o perioadă sau într-o orientare tematică}

\operatitle este o artă poetică, deoarece autorul își exprimă propriile convingeri despre arta literară, despre menirea literaturii, despre rolul artistului în societate.

\marginnote{artă poetică modernă}[0.3cm]
Este o artă poetică modernă, pentru că în cadrul ei apare o problematică specifică liricii moderne: transfigurarea socialului în estetic, estetica urâtului, raportul dintre inspirație și tehnica poetică. Interesul poeților de a reflecta asupra creației lor, de a-și sintetiza concepția artistică și de a o transmite cititorilor, astfel că fiecare volum este deschis de o artă poetică.

\section{Prezentarea imaginilor/ideilor poetice, relevante pentru tema și viziunea despre lume din textul studiat}

Tema poeziei o reprezintă creația literară în ipostaza de meșteșug, creație lăsată ca moștenire unui fiu spiritual -- posterității.

\marginnote{ipostaze lirice}[1.2cm]
Textul poetic este conceput ca un monolog adresat de tată unui fiu spiritual căruia îi este lăsată drept unică moștenire \textit{„cartea”}, metonimie care sugerează opera literară. Cele două ipostaze lirice sunt desemnate de pronumele \textit{eu} (tatăl spiritual, poetul) și \textit{tu} („fiul”, cititorul, urmașii), iar în finalul poeziei, de metonimiile \textit{„robul”} -- \textit{„Domnul”}. Lirismul subiectiv se justifică prin atitudinea poetică transmisă în mod direct și, la nivelul expresiei, prin mărcile eului liric: pronume și verbe de persoana \rom{1} singular (\textit{„eu am ivit”}, \textit{„am preschimbat”}), adjective posesive (\textit{„cartea mea”}), care se referă la eul liric, dar și pronume și verbe de persoana a \rom{2}-a singular (\textit{„te”}, \textit{„tine”}, \textit{„urci”}) sau substantive în vocativ (\textit{„fiule”}), care desemnează interlocutorul imaginar.

\marginnote{poet meșteșugar}[0.5cm]
Arta poetică \operatitle ilustrează viziunea lui Arghezi asupra lumii, atitudinea sa de poet responsabil în fața urmașilor cititori, responsabil pentru mesajul și valoarea estetică a operei sale. Este un poet angajat social, care își transfigurează în artă suferințele, apelând la \textit{estetica urâtului}: \textit{„Din bube, mucegaiuri și noroi/Iscat-am frumuseți și prețuri noi”}. Creatorul se proiectează în ipostaza poetului meșteșugar (\textit{poeta faber}), un \textit{„șlefuitor de cuvinte”}: \textit{„Din graiul lor cu-ndemnuri pentru vite/Eu am ivit cuvinte potrivite”}.

Poezia sa este o „carte”, adică un bun spiritual prin care scriitorul contribuie la emanciparea neamului său.


\section{Ilustrarea elementelor de compoziție și de limbaj ale textului poetic studiat, semnificative pentru tema și viziunea despre lume {\footnotesize (de exemplu: imaginar poetic, titlu, incipit, relații de opoziție și de simetrie, motiv poetic, laitmotiv, figuri semantce/tropi, elemente de prozodie etc.)}}

Titlul poeziei are două sensuri: unul denotativ și altul conotativ. În sens propriu, denotativ, cuvântul-titlu desemnează un act juridic prin care o persoană își exprimă dorințele ce urmează a-i fi îndeplinite după moarte, cu privire la transmiterea averii sale. Titlul amintește și de cărțile \textbf{\textit{Bibliei}}, \textbf{\textit{Vechiul Testament}} și \textbf{\textit{Noul Testament}}, care conțin învățături religioase adresate omenirii. De aici derivă sensul figurat, conotativ al titlului: creația argheziană, \textit{„cartea”}, este o moștenire spirituală lăsată de poet urmașilor.

\marginnote{cartea -- motiv central}[0.3cm]
Ca element de recurență, cuvântul \textit{„carte”} are o bogată serie sinonimică în text: \textit{„testament”}, \textit{„hrisov”}, \textit{„cuvinte potrivite”}, \textit{„slova de foc și slova făurită”}.

Textul poetic este structurat în cinci strofe cu număr inegal de versuri, grupate în trei secvențe poetice. Prima secvență (strofele \rom{1}, \rom{2}) sugerează legătura dintre generații: străbuni, poet și cititorii-urmași. Secvența a doua (strofele \rom{3})


\end{document}
