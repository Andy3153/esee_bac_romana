%%
%% Basic LaTeX preamble by Andy3153
%% created   04/10/21 ~ 18:21:32
%% modified1 15/10/21 ~ 23:15:05
%% modified2 03/11/21 ~ 18:37:45
%% modified3 22/02/22 ~ 12:48:17
%%
%% it used to be a template rip
%%
%% reguli de scriere:
%% 0. o recomandare mai mult, reține că am folosit xelatex, nu pdflatex
%%
%% 1. când apare [...] într-un citat, NU trebuie să fie italic și el
%%      exNU: \textit{„text1 [...] text2”}
%%      exDA: \textit{„text1} [...] \textit{text2”}
%%
%% 2. când apare o enumerație de citate, NU trebuie să fie italice și virgulele
%%      exNU: \textit{„citat1”, „citat2”, „citat3”}
%%      exDA: \textit{„citat1”}, \textit{„citat2”}, \textit{„citat3”}
%%
%% 3. să NU-ți fie frică să folosești \hbox dacă îti desparte latex aiurea cuvinte cu cratimă
%%
%% 4. să NU-ți fie frică să folosești \- ca să corectezi dacă îți desparte latex aiurea în silabe (destul de rar dacă folosești xelatex, de ce nu știu am luat-o ca atare, probabil ca are utf-8 by default și de aia)
%%
%% 5. în sectiunile care au un text în paranteză unde da exemple, formatează exemplul ăla astfel:
%%      ex: \section{text1 {\footnotesize\normalfont (de exemplu: text2, text3 etc.)}}
%%
%%
%%
%% 69.0. de șters despărțirile în silabe degeaba de când foloseam pdflatex
%%
%% 69.1. de convertit numerele romane în \rom{numar}
%%
%% 69.2. de convertit in preambul \operatitle in \textbfit
%%
%% 69.3. de convertit în text \opera{title,author} în \opera{title,author}\ unde e nevoie
%%
%% 69.4. de pus 2 enteruri la început se secțiune/subsecțiune nouă
%%
%% 69.5. de facut aia cu footnoteize in sectiuni
%%


\documentclass[
 12pt,                        % Font size
 a4paper                      % Paper type
]{article}


% Packages
\usepackage[
 margin=2.7cm,                % Margin size
 marginparwidth=2cm,          % Margin note size
 marginparsep=3mm             % Space between margin and text
]{geometry}
%\usepackage[utf8]{inputenc} % UTF-8 support, disabled because of switch to XeLaTeX
%\usepackage[T1]{fontenc}    % Proper hyphenation, disabled because of switch to XeLaTeX
\usepackage[romanian]{babel} % Romanian characters support
\usepackage{indentfirst}     % Add paragraph indentation even after a section
\usepackage{marginnote}      % Notes on the margins of a document (more advanced \marginpar)
\usepackage{titlesec}        % Customize titles


% Custom format for titles, sections, subsections etc.
\titleformat*{\section}{\large\bfseries}
\titleformat{\subsection}{\normalfont\normalfont\bfseries}{}{1.5em}{}


% Page numbering
%\pagenumbering{gobble} % uncomment if you want to disable it


% Custom commands
% Format: \newcommand{\command}[variable]{action #variable}
\newcommand{\rom}[1]{\uppercase\expandafter{\romannumeral #1\relax}} % Roman numerals
\newcommand{\textbfit}[1]{\textbf{\textit{#1}}}                      % combine bold and italic
\newcommand{\operatitle}{}                                           % to not get errors
\newcommand{\operaauthor}{}                                          % to not get errors



% Customize \marginnote font
\renewcommand\marginfont{\ttfamily\footnotesize}


% Make \ttfamily hyphenate words for the margin notes
\DeclareFontFamily{OT1}{cmtt}{\hyphenchar\font=-1}
\DeclareFontFamily{\encodingdefault}{\ttdefault}{\hyphenchar\font=`\-}
\DeclareFontFamily{T1}{cmtt}{\hyphenchar\font=45}


% Make subsections not appear in ToC as we're using them for a completely different thing
\setcounter{tocdepth}{1}


% Basic document info
\date{}   % Show no date in the title
\author{} % Empty author to not get a warn about missing author
