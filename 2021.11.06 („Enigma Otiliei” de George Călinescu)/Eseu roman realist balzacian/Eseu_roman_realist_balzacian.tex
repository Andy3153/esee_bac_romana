% Include the preamble
%%
%% Basic LaTeX preamble by Andy3153
%% created   04/10/21 ~ 18:21:32
%% modified1 15/10/21 ~ 23:15:05
%% modified2 03/11/21 ~ 18:37:45
%% modified3 22/02/22 ~ 12:48:17
%%
%% it used to be a template rip
%%
%% reguli de scriere:
%% 0. o recomandare mai mult, reține că am folosit xelatex, nu pdflatex
%%
%% 1. când apare [...] într-un citat, NU trebuie să fie italic și el
%%      exNU: \textit{„text1 [...] text2”}
%%      exDA: \textit{„text1} [...] \textit{text2”}
%%
%% 2. când apare o enumerație de citate, NU trebuie să fie italice și virgulele
%%      exNU: \textit{„citat1”, „citat2”, „citat3”}
%%      exDA: \textit{„citat1”}, \textit{„citat2”}, \textit{„citat3”}
%%
%% 3. să NU-ți fie frică să folosești \hbox dacă îti desparte latex aiurea cuvinte cu cratimă
%%
%% 4. să NU-ți fie frică să folosești \- ca să corectezi dacă îți desparte latex aiurea în silabe (destul de rar dacă folosești xelatex, de ce nu știu am luat-o ca atare, probabil ca are utf-8 by default și de aia)
%%
%% 5. în sectiunile care au un text în paranteză unde da exemple, formatează exemplul ăla astfel:
%%      ex: \section{text1 {\footnotesize\normalfont (de exemplu: text2, text3 etc.)}}
%%
%%
%%
%% 69.0. de șters despărțirile în silabe degeaba de când foloseam pdflatex
%%
%% 69.1. de convertit numerele romane în \rom{numar}
%%
%% 69.2. de convertit in preambul \operatitle in \textbfit
%%
%% 69.3. de convertit în text \opera{title,author} în \opera{title,author}\ unde e nevoie
%%
%% 69.4. de pus 2 enteruri la început se secțiune/subsecțiune nouă
%%
%% 69.5. de facut aia cu footnoteize in sectiuni
%%


\documentclass[
 12pt,                        % Font size
 a4paper                      % Paper type
]{article}


% Packages
\usepackage[
 margin=2.7cm,                % Margin size
 marginparwidth=2cm,          % Margin note size
 marginparsep=3mm             % Space between margin and text
]{geometry}
%\usepackage[utf8]{inputenc} % UTF-8 support, disabled because of switch to XeLaTeX
%\usepackage[T1]{fontenc}    % Proper hyphenation, disabled because of switch to XeLaTeX
\usepackage[romanian]{babel} % Romanian characters support
\usepackage{indentfirst}     % Add paragraph indentation even after a section
\usepackage{marginnote}      % Notes on the margins of a document (more advanced \marginpar)
\usepackage{titlesec}        % Customize titles
%\usepackage{ulem}            % More underlines
\usepackage[dash,dot]{dashundergaps}


% Custom format for titles, sections, subsections etc.
\titleformat*{\section}{\large\bfseries}
\titleformat{\subsection}{\normalfont\normalfont\bfseries}{}{1.5em}{}


% Page numbering
%\pagenumbering{gobble} % uncomment if you want to disable it


% Custom commands
% Format: \newcommand{\command}[variable]{action #variable}
\newcommand{\rom}[1]{\uppercase\expandafter{\romannumeral #1\relax}} % Roman numerals
\newcommand{\textbfit}[1]{\textbf{\textit{#1}}}                      % combine bold and italic
\newcommand{\ex}[1]{\dashuline{\texttt{\footnotesize#1}}}                              % things to be filled by the reader
\newcommand{\comment}[1]{}                                           % comments
\newcommand{\operatitle}{}                                           % to not get errors
\newcommand{\operaauthor}{}                                          % to not get errors
\newcommand{\ApplySubIIStyling}                                      % custom styling for a part of the document
{
 \setcounter{tocdepth}{1} % Now begin allowing sections to appear
 %\newcommand{\sectionbreak}{\ifnum\value{section}>0 \clearpage\fi}
}


% Customize \marginnote font
\renewcommand\marginfont{\ttfamily\footnotesize}


% Make \ttfamily hyphenate words for the margin notes
\DeclareFontFamily{OT1}{cmtt}{\hyphenchar\font=-1}
\DeclareFontFamily{\encodingdefault}{\ttdefault}{\hyphenchar\font=`\-}
\DeclareFontFamily{T1}{cmtt}{\hyphenchar\font=45}


% Make subsections not appear in ToC as we're using them for a completely different thing
\setcounter{tocdepth}{1}


% Basic document info
\date{}   % Show no date in the title
\author{} % Empty author to not get a warn about missing author


\newcommand{\operatitle}{\textbfit{„Enigma Otiliei”}} % title of the text
\newcommand{\operaauthor}{George Călinescu} % author of the text

\title{Eseu privind tema și viziunea despre lume într-un roman realist-balzacian studiat}


\begin{document}
\maketitle % Show the title
%\reversemarginpar % put margin notes on left instead of on right

% Beginning of text


\subsection{Context}

Publicat în 1938, romanul \operatitle\ apare la sfârșitul perioadei interbelice, și este al doilea dintre cele patru romane scrise de \operaauthor.


\section{Evidențierea trăsăturilor care fac posibilă încadrarea romanului studiat într-o tipologie, într-un curent cultural/literar, într-o orientare tematică}

\marginnote{specie literară}[0.3cm]
Opera literară \operatitle\ este un roman deoarece are o acțiune amplă, desfășurată pe mai multe planuri, cu un conflict complex, la care participă numeroase personaje.

\marginnote{ti\-po\-lo\-gi\-e/curente literare}[0.5cm]
Este roman realist de tip balzacian deoarece apar tema familiei și motivul moștenirii și al paternității, pentru că structura este închisă, iar relatarea întâmplărilor este făcută la persoana a \rom{3}-a , din perspectiva unui narator omniscient, omniprezent și obiectiv; este utilizată tehnica detaliului semnificativ, iar personajele sunt prezentate în relație cu mediul din care provin, fiind tipice pentru o anumită categorie socială. Cu toate acestea, romanul depășește modelul realismului clasic, prin spiritul critic și polemic, prin elementele ce țin de modernitate, precum ambiguitatea personajelor.

\marginnote{tehnici moderne}[0.3cm]
Portretul Otiliei este realizat prin tehnici moderne: comportamentismul și reflectarea poliedrică (pluriperspectivismul). Până în capitolul al \rom{16}-lea, Otilia este prezentată mai ales prin comportamentism (fapte, gesturi, replici),
\marginnote{comportamentismul și reflectarea poliedrică}[0.6cm]
fără a-i cunoaște gândurile din perspectiva unică a naratorului. Această tehnică este dublată de reflectarea personalității Otiliei în conștiința celorlalte personaje, ceea ce îi alcătuiește un portret complex și contradictoriu: este \textit{„fe-fetița”} cuminte și iubitoare pentru moș Costache, fata exuberantă, \textit{„admirabilă, superioară”} pentru Felix, femeia capricioasă, \textit{„cu un temperament de artistă”} pentru Pascalopol, \textit{„o dezmățată, o stricată”} pentru Aglae, \textit{„o fată deșteaptă”}, cu spirit practic, pentru Stănică, o rivală în căsătorie pentru Aurica.

\marginnote{naturalismul}[0.3cm]
Un alt aspect modern, naturalismul, constă în interesul pentru procesele psihice deviante, alienarea și senilitatea, motivate prin ereditate și mediu. Titi, fiul retardat care se îndreaptă spre demență, este o copie a tatălui său, Simion Tulea. Aurica, fata bătrână, invidioasă și rea, este o copie degradată a mamei. Amândouă au preocupări obsesive: Aglae -- moștenirea, Aurica -- dorința de a se căsători. Universul familiei Tulea se află sub semnul bolii, al degradării morale reflectate în plan fizic.


\section{Ilustrarea temei romanului în episoade/citate/secvențe comentate}

\marginnote{roman balzacian și citadin}[0.3cm]
Prin temă, romanul este balzacian și citadin. Caracterul citadin ține de modernismul lovinescian. Imaginea societății constituie fundalul pe care se proiectează formarea/maturizarea tânărului Felix Sima, care, înainte de a-și face o carieră, trăiește experiența iubirii și a relațiilor de familie.


\section{Prezentarea elementelor de structură și de compoziție ale textului narativ, semnificative pentru tema și viziunea despre lume din romanul studiat {\footnotesize\normalfont(de exemplu: acțiune, conflict, relații temporale și spațiale, incipit, final, tehnici narative, perspectivă narativă, registre stilistice, limbajul personajelor etc.)}}

\marginnote{ideea balzaciană a paternității}[-0.4cm]
Titlul inițial, \textbfit{„Părinții Otiliei”}, reflectă ideea balzaciană a paternității, deoarece fiecare dintre personaje determină cumva soarta orfanei Otilia, ca niște „părinți”. Autorul schimbă titlul din motive editoriale și deplasează accentul de la motivul realist al paternității la misterul protagonistei.

\marginnote{perspectiva narativă obiectivă}[-0.2cm]
Întâmplările din roman sunt relatate la persoana a \rom{3}-a, din perspectiva unui narator omniscient și omniprezent.

\marginnote{compoziție}[1.3cm]
Romanul, alcătuit din douăzeci de capitole, este construit pe mai multe planuri narative, care urmăresc destinul unor personaje: destinul Otiliei, al lui Felix, al membrilor clanului Tulea, al lui Stănică etc. Un plan urmărește lupta dusă de clanul Tulea pentru obținerea moștenirii lui Costache Giurgiuveanu și înlăturarea Otiliei Mărculescu. Al doilea plan prezintă destinul tânărului Felix Sima care, rămas orfan, vine la București pentru a studia Medicina, locuiește în casa tutorelui său, avarul Costache Giurgiuveanu și trăiește iubirea adolescentină pentru Otilia.

\marginnote{incipitul fixează cadrul temporal și spațial}[-0.6cm]
Incipitul romanului realist fixează veridic cadrul temporal (\textit{„Într-o seară de la începutul lui iulie 1909”}) și spațial (strada Antim din București, casa lui moș Costache), prezintă principalele personaje, sugerează conflictul și trasează principalele planuri epice. Finalul este închis prin rezolvarea conflictului și este urmat de un epilog.

\marginnote{simetria incipitului cu finalul}[0.1cm]
Simetria incipitului cu finalul se realizează prin descrierea străzii și a casei lui moș Costache.

\marginnote{conflicte}[1.3cm]
Istoria moștenirii include un dublu conflict succesoral: este vorba pe de o parte, de ostilitatea manifestată de Aglae împotriva orfanei Otilia, și pe de altă parte, de interesul lui Stănică pentru averea bătrânului, care duce la dezbinarea familiei Tulea. Conflictul erotic privește rivalitatea dintre adolescentul Felix și maturul Pascalopol pentru iubirea Otiliei.

În conflictul pentru moștenire se află două familii înrudite. Membrii acestora aparțin unor tipologii umane care conturează universul social. În casa lui Costache Giurgiuveanu, proprietarul averii, trăiește Otilia Mărculescu, adolescentă orfană, fiica celei de-a doua soții decedate a acestuia. Aici ajunge orfanul Felix Sima, venit la tutorele său din București pentru a studia Medicina. Moșierul Leonida Pascalopol, prieten al bătrânului vine în casă din dorința de a avea o familie și din afecțiune pentru Otilia, pe care o cunoaște de mică.

În casa vecină trăiește o a doua familie, înrudită cu prima, care aspiră la moștenirea averii bătrânului. „Clanul” Tulea este condus de sora lui Costache, Aglae. Din familie fac parte soțul acesteia, Simion Tulea și cei trei copii ai lor, Olimpia, Aurica și Titi. Acestei familii i se adaugă Stănică Rațiu, soțul Olimpiei, dornic să obțină moștenirea.

\marginnote{expozițiunea realizată după metoda realist balzaciană}[-0.4cm]
Expozițiunea este realizată în metoda realist-balzaciană: situarea exactă a acțiunii în timp și spațiu, veridicitatea susținută prin detaliile topografice, finețea observației și notarea detaliului semnificativ. Caracteristicile arhitectonice ale străzii și ale casei lui moș Costache sunt surprinse de \textit{„ochiul unui estet”}, din perspectiva naratorului specializat, deși observația îi este atribuită personajului-reflector, Felix Sima, care caută o anumită casă.

Odată intrat în casă, Felix îi cunoaște pe unchiul său și pe verișoara Otilia, apoi asistă la o scenă de familie: jocul de table. Naratorul îi atribuie lui Felix observarea obiectivă a personajelor prezente în odaia înaltă în care este introdus.

\marginnote{intriga se dezvoltă pe două planuri}[-0.05cm]
Intriga se dezvoltă pe două planuri care se întrepătrund: pe de o parte, este prezentată istoria moștenirii lui Costache Giurgiuveanu, iar pe de altă parte, romanul are în centru destinul tânărului Felix Sima, maturizarea lui (ceea ce-i conferă cărții caracterul de Bildungsroman).

Lupta pentru moștenirea bătrânului avar este un prilej pentru observarea efectelor, în plan moral, ale obsesiei banului. Moș Costache, proprietar de imobile, restaurante și acțiuni, nutrește iluzia longevității și nu pune în practică niciun proiect pentru a-i asigura viitorul Otiliei. Clanul Tulea urmărește să moștenească averea lui, plan periclitat ipotetic de adopția Otiliei. Deși are o afecțiune sinceră pentru fată, bătrânul amână adopția ei, de dragul banilor și pentru că se teme de Aglae. El încearcă totuși să pună în aplicare niște planuri pentru a o proteja pe Otilia, intenționând să-i construiască o casă cu materiale provenite de la demolări. Proiectele lui moș Costache nu se realizează, deoarece, din cauza efortului depus la transportarea materialelor, bătrânul este lovit de o criză de apoplexie. Chiar dacă pentru familia Tulea boala lui moș Costache reprezintă un prilej de a-i ocupa militărește casa, în așteptarea morții bătrânului și a obținerii moștenirii, îngrijirile lui Felix, ale Otiliei și ale lui Pascalopol determină însănătoșirea bătrânului avar. Moartea lui moș Costache este provocată, în cele din urmă, de Stănică Rațiu, ginerele Aglaei, care urmărește să parvină și îi fură avarului banii de sub saltea. În deznodământ, Olimpia e părăsită de Stănică, Aurica nu-și poate face o situație, iar Felix o pierde pe Otilia.

Alături de avariție, lăcomie și parvenitism, sunt înfățișate alte aspecte ale familiei burgheze: relația dintre părinți și copii, căsătoria, condiția orfanului.

Planul formării tânărului Felix, student la Medicină, urmărește experiențele trăite de acesta în casa unchiului său, în special iubirea adolescentină pentru Otilia. Este gelos pe Pascalopol, dar nu ia nicio decizie, fiindcă primează dorința de a-și face o carieră. Otilia îl iubește pe Felix, dar după moartea lui moș Costache îl părăsește, considerând că reprezintă o piedică în calea realizării lui profesionale. Ea se căsătorește cu Pascalopol, bărbat matur, care îi poate oferi înțelegere și protecție. În epilog, aflăm că, generos, Pascalopol i-a redat Otiliei libertatea de a-și trăi tinerețea, ea devenind soția unui conte exotic și căzând în \textit{„platitudine”}. Otilia rămâne pentru Felix o imagine a eternului feminin, iar pentru Pascalopol, o \textit{„enigmă”}.

Pentru portretizarea personajelor, autorul alege tehnica balzaciană a descrierii mediului și fizionomiei, din care se pot deduce trăsăturile de caracter. Portretul balzacian pornește de la caracterele clasice (avarul, ipohondrul, gelosul, ambițiosul, cocheta), \hbox{cărora} \marginnote{tehnici balzaciene de portretizare a personajelor}[-0.1cm]
realismul le conferă dimensiune socială și psihologică, adăugând un alt tip uman, arivistul. Tendința de generalizare conduce la realizarea unei tipologii clasice: moș Costache - avarul, Aglae -- \textit{„baba absolută fără cusur în rău”}, \hbox{Aurica -- fa}ta bătrână, \hbox{Simion -- de}mentul senil, Titi -- debilul mintal, Stănică Rațiu -- arivistul, \hbox{Otilia -- co}cheta, Felix -- ambițiosul, Pascalopol -- aristocratul rafinat.


\subsection{Concluzie}

Roman al unei familii și istorie a unei moșteniri, \operatitle\ de \operaauthor\ se încadrează în categoria prozei realist-balzaciene, deși criticul N. Manolescu consideră că este de un \textit{„balzacianism fără Balzac”}.
\end{document}
