\section{Perspectiva narativă obiectivă}

Perspectiva narativă este punctul de vedere al naratorului, unghiul din care privește și interpretează faptele.

În fragmentul supus analizei, extras din opera literară \ex{titlul} de \ex{autorul}, perspectiva narativă este obiectivă și omniscientă, cu focalizare zero și viziune „dindărăt”.

Obiectivitatea este evidențiată prin folosirea formelor verbale și pronominale de persoana a \rom{3}-a (\ex{3 verbe, 3 pronume}), realizată din unghiul unui narator ce povestește din postura unui observator extern, neimplicat în acțiune. Acesta relatează totul pe un ton neutru, impersonal. Omnisciența externă este dovedită de faptul că naratorul știe tot ce se întâmplă.

Astfel, prin intermediul său, se relatează cum \ex{scurtă povestire}. Omnisciența internă este dovedită de faptul că naratorul știe ce gândesc și ce simt personajele: este surprinsă \ex{ce?}

Astfel, obiectivitatea perspectivei narative conferă un caracter de veridicitate evenimentelor relatate.
