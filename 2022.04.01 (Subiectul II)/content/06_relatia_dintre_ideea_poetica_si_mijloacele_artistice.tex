\section{Relația dintre ideea poetică și mijloacele artistice}

Ideea poetică reprezintă mesajul central identificat într-o operă literară. Mijloacele artistice surprind totalitatea mecanismelor utilizate în consolidarea temei și a ideii poetice.

Versurile dare spre analiză redau tema \ex{ex: trecerii ireversibile a timpului}, așa cum se reflectă în existența umană, aflată în raport cu \ex{ex: trecerea timpului}.

Astfel, \ex{figuri de stil + exemple} redau/sugerează ideea \ex{ex: infimității omului}, în raport cu \ex{ex: universul}

La nivel morfologic, ideea poetică este susținută de verbele la timpul \ex{ce timp?}: \ex{3 verbe}, care descriu/redau \ex{ex: încheierea zilei, peisajul, ce se întâmplă}.

Prozodia imprimă întregii poezii ritmul interior al sufletului eului liric, susținând ideea \ex{ex: infimității umane, trecerii timpului}, prin măsura de \ex{câte?} silabe, rima \ex{de care?} și ritmul \ex{de care?}.

Așadar, ideea poetică se află în strânsă legătură cu mijloacele artistice, accentuând \ex{ex: tristețea eului liric}, provocată de \ex{ex: trecerea timpului}.
