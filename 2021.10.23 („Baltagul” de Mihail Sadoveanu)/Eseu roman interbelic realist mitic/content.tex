% Commands
\renewcommand{\operatitle}{\textbfit{„Baltagul”}} % title of the text
\renewcommand{\operaauthor}{Mihail Sadoveanu} % author of the text


% Beginning of text
\subsection{Context}

Opera \operatitle\ de \operaauthor, publicată în 1930, este un roman interbelic, obiectiv, realist-mitic și tradițional. Structura polimorfă este dată de \textit{„amestecul de roman realist și narațiune arhetipală grefată pe un scenariu polițist”}.


\section{Evidențierea trăsăturilor care fac posibilă încadrarea romanului studiat într-o tipologie, într-un curent cultural/literar, într-o orientare tematică}

Romanul este o creație epică în proză, de mari dimensiuni, cu acțiune complexă, desfășurată pe mai multe planuri, în timp și spațiu precizate, antrenând un număr mare de personaje puternic individualizate.

\marginnote{aspectul realist}[0.3cm]
Romanul este realizat pe două coordonate fundamentale: aspectul realist (monografia lumii pastorale, reperele spațiale, tipologia personajelor, tehnica detaliului semnificativ) și aspectul mitic (gesturile rituale ale Vitoriei, tradițiile pastorale, motivul comuniunii om-natură și mitul marii treceri).

Inspirația romanului din balada populară \textbfit{„Miorița”}, sugerată chiar de scriitor prin mottoul \textit{„Stăpâne, stăpâne, / Mai chiamă ș-un câne...”}, constituie un aspect controversat în receptarea critică, susținut de criticii interbelici (E. Lovinescu, G. Călinescu) și adesea contestat de exegeza postbelică (Al. Paleologu).


\section{Prezentarea temelor romanului studiat}

Marile teme sadoveniene se regăsesc aici: viața pastorală, natura, călătoria, miturile, iubirea, familia, arta povestirii, înțelepciunea.

\marginnote{tema}[0.3cm]
Tema rurală a romanului tradițional este dublată de tema călătoriei inițiatice și justițiare. Romanul \operatitle\ prezintă monografia satului moldovenesc de la munte, lumea arhaică a păstorilor, având în prim-plan căutarea și pedepsirea celor care l-au ucis pe Nechifor Lipan. Însoțită de Gheorghiță, Vitoria reconstituie drumul parcurs de bărbatul său, pentru descoperirea adevărului și realizarea dreptății.

\marginnote{tema labirintului}[0.3cm]
Călătoria, căutarea adevărului constituie axul romanului și se asociază cu motivul labirintului. Parcurgerea drumului are diferite semnificații. Vitoria reconstituie evenimentele care au condus la moartea bărbatului ei (intriga polițistă), ceea ce devine o dublă aventură: a cunoașterii lumii și a cunoașterii de sine. Pentru Gheorghiță, călătoria are rol educativ, de inițiere a tânărului (\textit{Bildungsroman}). Nechifor, personaj absent, prezentat indirect, aparține planului mitic. Căutându-și soțul, Vitoria parcurge simultan două lumi: spațiul real, concret și comercial și o lume „de semne și minuni”, al căror sens ea știe să-l descifreze.


\section{Prezentarea elementelor de structură și de compoziție ale textului narativ, semnificative pentru tema și viziunea despre lume din romanul studiat}% {\footnotesize\normalfont (de exemplu: acțiune, conflict, perspectivă narativă, registre stilistice, limbajul personajelor etc.)}}

\marginnote{motivul labirintului}[0.8cm]
Motivul labirintului se concretizează la nivelul acțiunii, dar este semnificativ și la nivelul titlului. Baltagul (toporul cu două tăișuri) este un obiect simbolic, ambivalent: armă a crimei și instrumentul actului justițiar, reparator. De remarcat că în roman, același baltag (al lui Lipan) îndeplinește cele două funcții. Baltagul tânărului Gheorghiță se păstrează neatins de sângele ucigașilor.

\marginnote{perspectiva obiectivă}[-0.1cm]
Narațiunea se face la persoana a \rom{3}-a, iar naratorul omniprezent și omniscient reconstituie lumea satului de munteni și acțiunile Vitoriei, în mod obiectiv, prin tehnica detaliului și observație.

\marginnote{timpul}[0.3cm]
Timpul derulării acțiunii este vag precizat, prin repere temporale din calendarul religios al satului tradițional: \textit{„aproape de Sf. Andrei”}, \textit{„în Postul Mare”}, \textit{„10 Martie”}.
\marginnote{cadrul}[0.3cm]
Cadrul acțiunii este satul de munte Măgura Tarcăului, zona Dornelor și a Bistriței, dar și satul de câmpie, Cristești, în Baia Jijiei.

\marginnote{compoziție}[0.3cm]
Romanul este structurat în șaisprezece capitole cu acțiune desfășurată cronologic, urmărind momentele subiectului. În raport cu tema călătoriei, capitolele pot fi grupate în trei părți: \rom{1}. constatarea absenței și pregătirile de drum, \rom{2}. căutarea soțului dispărut, \rom{3}. găsirea celui căutat, înmormântarea și pedepsirea făptașilor.

\marginnote{incipitul}[0.3cm]
Incipitul romanului este o legendă despre ocupațiile și modul de viață al păstorilor și al altor neamuri, pe care o spunea Nechifor la \textit{„cumătrii și nunți”}. Legenda este rememorată de Vitoria în absența soțului ei și anticipează destinul acestuia, având rol de prolog.
\marginnote{finalul}[0.3cm]
Finalul (epilogul) cuprinde planurile de viitor ale Vitoriei în legătură cu familia ei, rostite după încheierea deznodământului.

Prima parte (capitolele \rom{1} -- al \rom{6}-lea) -- frământările Vitoriei în așteptarea soțului și pregătirile de drum -- include expozițiunea și intriga.

\marginnote{expozițiunea}[0.3cm]
În expozițiune se prezintă satul Măgura Tarcăului și schița portretului fizic al Vitoriei, care toarce pe prispă, gândindu-se la întârzierea soțului său plecat la Dorna să cumpere oi.

\marginnote{intriga}[0.8cm]
Intriga cuprinde frământările ei, dar și acțiunile întreprinse înainte de plecarea în căutarea soțului: ține post negru douăsprezece vineri, se închină la icoana Sfintei Ana de la Mânăstirea Bistrița, anunță autoritățile dispariția soțului, vinde unele lucruri pentru a face rost de bani de drum, pe Minodora o lasă la Mânăstirea Văratec, iar lui Gheorghiță îi încredințează un baltag sfințit.

\marginnote{desfășurarea acțiunii}[0.8cm]
Partea a doua (capitolele al \rom{7}-lea -- al \rom{13}-lea) conține desfășurarea acțiunii și prezintă drumul parcurs de Vitoria și ale fiului ei, Gheorghiță, în căutarea lui Nechifor Lipan. Ei reconstituie traseul lui Nechifor, făcând o serie de popasuri: la hanul lui Donea de la gura Bicazului, la crâșma domnului David de la Călugăreni, la moș Pricop și baba Dochia din Fărcașa, la Vatra Dornei, apoi spre Păltiniș, Broșteni, Borca, de unde drumul părăsește apa Bistriței, \textit{„într-o țară cu totul necunoscută”}. De asemenea, întâlnesc botez și nuntă -- marile momente din viața omului, a căror ordine sugerează Vitoriei înmormântarea din final.

Întrebând din sat în sat, ea își dă seama că soțul său a dispărut între Suha și Sabasa. Cu ajutorul câinelui regăsit, Lupu, munteanca descoperă rămășițele lui Lipan într-o râpă, în dreptul Crucii Talienilor.

Partea a treia (capitolele al \rom{14}-lea -- al \rom{16}-lea) prezintă sfârșitul drumului: ancheta poliției, înmormântarea, parastasul lui Nechifor Lipan și pedepsirea ucigașilor.

Coborârea în râpă și veghea nocturnă a mortului sunt probe de maturizare pentru Gheorghiță, încheiată cu înfăptuirea actului de dreptate la parastas.

\marginnote{punctul culminant}[0.3cm]
Punctul culminant este scena de la parastas, în care Vitoria povestește cu fidelitate scena crimei, surprinzându-i și pe ucigașii Ilie Cuțui și Calistrat Bogza. Primul își recunoaște vina, însă al doilea devine agresiv. Este lovit de Gheorghiță cu baltagul lui Nechifor și sfâșiat de câinele Lupu, făcându-se astfel dreptate.

În deznodământ, ucigașul Calistrat Bogza îi cere iertare și-și recunoaște fapta.

\marginnote{personajele}[0cm]
Personajele înfățișează tipologii umane reprezentative pentru lumea satului de la munte, la începutul secolului al \rom{20}-lea.

\marginnote{personajul principal}[0.3cm]
Personajul principal este Vitoria Lipan, femeia voluntară, munteanca, soție de cioban. Vitoria este o femeie puternică, hotărâtă, curajoasă, lucidă. Inteligența nativă și stăpânirea de sine sunt evidențiate pe drum, dar mai ales la parastas, când demască ucigașii.

Aparținând lumii arhaice, patriarhale, Vitoria transmite copiilor respectul tradițiilor și este refractară la noutățile civilizației: \textit{„În tren ești olog, mut și chior”}. Ca mamă, îi interzice Minodorei să se îndepărteze de tradiție (\textit{„Îți arăt eu coc, valț și bluză...! Nici eu, nici bunică-ta, nici bunică-mea n-am știut de acestea -- și-n legea noastră trebuie să trăiești și tu!”}) și contribuie prin călătorie la maturizarea lui Gheorghiță. Respectă obiceiurile de cumetrie, de nuntă și de înmormântare.

\marginnote{tehnica basoreliefului}[-0.1cm]
Personajul complex este realizat prin tehnica basoreliefului și individualizat prin caracterizare directă și indirectă. Portretul fizic relevă frumusețea personajului prin tehnica detaliului semnificativ.

\marginnote{personajul secundar}[0.2cm]
Personajul secundar, Gheorghiță, reprezintă generația tânără, care trebuie să ia locul tatălui dispărut. Romanul poate fi considerat inițiatic, deoarece prezintă drumul spre maturizarea lui Gheorghiță.

Nechifor Lipan este caracterizat în absență, prin retrospectivă și remorare, și simbolizează destinul muritor al oamenilor.

\marginnote{personaje episodice}[0.3cm]
Personaje episodice sunt reprezentative pentru lumea satului arhaic: Minodora, fiica receptivă la noutățile civilizației, este trimisă la mânăstire pentru purificare; moș Pricop (ospitalitatea), părintele Dănilă (autoritatea spirituală în satul arhaic), baba Maranda (credința în superstiții).

Romanul are caracter monografic deoarece înfățișează viața muntenilor, ocupațiile, tradițiile, obiceiurile și principalele lor trăsături: muncitori, veseli, iubitori.
\marginnote{personajul colectiv}[0.1cm]
Personajul colectiv, muntenii este portretizat încă de la început, în legenda pe care obișnuia să o spună Lipan la cumetrii, dar și în mod direct, de către narator.

Ca moduri de expunere, narațiunea preponderentă este nuanțată de secvențele dialogate, iar pasajele descriptive fixează diferite aspecte ale cadrului sau elemente de portret fizic, individual și colectiv.


\subsection{Concluzie}
În concluzie, romanul \operatitle\ de \operaauthor\ aparține realismului mitic.
