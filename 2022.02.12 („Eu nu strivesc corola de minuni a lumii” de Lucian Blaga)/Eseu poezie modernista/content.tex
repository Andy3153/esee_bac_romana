% Commands
\renewcommand{\operatitle}{\textbfit{„Eu nu strivesc corola de minuni a lumii”}} % title of the text
\renewcommand{\operaauthor}{Lucian Blaga} % author of the text


% Beginning of text
\subsection{Context}

Poezia \operatitle\ de \operaauthor, publicată în deschiderea primului său volum, \textbf{\textit{„Poemele luminii”}} în 1919, face parte din seria artelor poetice moderne ale literaturii române interbelice, alături de \textbf{\textit{„Testament”}} de Tudor Arghezi și \textbf{\textit{„Joc secund”}} de Ion Barbu.


\section{Evidențierea trăsăturilor care fac posibilă încadrarea textului poetic studiat într-un curent cultural/literar, într-o perioadă sau într-o orientare tematică}

Este o artă poetică deoarece autorul își exprimă aici, prin mijloace artistice, concepția despre poezie (teme, modalități de creație și de expresie) și despre rolul poetului, din perspectiva unei estetici moderne. Este o meditație filosofică și o confesiune elegiacă pe tema cunoașterii.

\marginnote{artă poetică modernă}[0.3cm]
Este o artă poetică modernă prin: influențele expresioniste, intelectualizarea emoției, noutatea metaforei (metaforă plasticizantă și metaforă revelatorie), înnoirea prozodiei (versul liber și ingambamentul). Poezia ilustrează unele dintre influențele expresioniste pe care le aduce în peisajul literar al vremii volumul \textbf{\textit{„Poemele luminii”}}: exacerbarea eului creator, sentimentul absolutului, interiorizarea și spiritualizarea peisajului, tensiunea lirică.


\section{Prezentarea imaginilor/ideilor poetice, relevante pentru te\-ma și viziunea despre lume din textul studiat}

Tema poeziei este cunoașterea, desemnată de metafora \textit{„lumina”}, dar și atitudinea poetică în fața marilor taine ale Universului. Cunoașterea lumii în planul creației poetice este posibilă numai prin iubire: \textit{„Eu nu strivesc.../căci eu iubesc/și flori și ochi și buze și morminte”}.

Viziunea despre lume se circumscrie orizontului misterului, un concept central la Blaga, în opera filosofică și în cea poetică. Între filosofia și poezia lui Lucian Blaga există o circulație a conceptelor.
\marginnote{viziunea despre lume}[0.3cm]
Pentru filosof, există două modalități de cunoaștere, de raportare la mister: cunoașterea luciferică sau minus cunoașterea, care potențează misterul, și cea paradiziacă sau plus cunoașterea, care descifrează misterul. Cunoașterea se articulează prin două tipuri de metafore: plasticizante și revelatorii.

Concepția poetului despre cunoaștere este exprimată artistic prin opoziția dintre metaforele revelatorii \textit{„lumina altora”} (cunoașterea paradiziacă, de tip rațional, logic) și \textit{„lumina mea”} (cunoașterea luciferică, poetică, de tip intuitiv). Rolul poetului nu este de a descifra tainele lumii și de a le ucide astfel (\textit{„Lumina altora/sugrumă vraja nepătrunsului ascuns”}), ci de a le adânci misterul prin creație: \textit{„eu cu lumina mea sporesc a lumii taină”}. Astfel, creația poetică apare ca un mijlocitor între eu și lume.


\section{Ilustrarea elementelor de compoziție și de limbaj ale textului poetic, semnificative pentru tema și viziunea despre lume {\footnotesize\normalfont(de exemplu: imaginar poetic, titlu, incipit, relații de opoziție și de simetrie, motiv poetic, laitmotiv, figuri semantice/tropi, elemente de prozodie etc.)}}

Fiind o poezie de tip confesiune, lirismul subiectiv se exprimă prin mărcile eului liric: verbe și pronume la persoana \rom{1} singular (\textit{„eu iubesc”}, \textit{„nu strivesc”}, \textit{„nu ucid”}), adjectivul posesiv de persoana \rom{1} (\textit{„calea mea”}, \textit{„ochii mei”}), topica și pauza afectivă/cenzura.

\marginnote{semnificația titlului}[0.3cm]
Titlul conține o metaforă revelatorie (\textit{„corola de minuni a lumii”}) și exprimă atitudinea poetului, izvorâtă din iubire, de protejare a misterului. Primul cuvânt, pronumele \textit{„eu”}, reluat de cinci ori în poezie, evidențiază rolul eului liric, (exacerbarea eului creator) și o marcă a confesiunii. Verbul la forma negativă \textit{„nu strivesc”} exprimă refuzul cunoașterii de tip rațional și opțiunea pentru cunoașterea luciferică.

Titlul este reluat în incipitul poeziei, iar sensul său, îmbogățit prin seria de antiteze și de metafore, se întregește cu versurile finale: \textit{„Eu nu strivesc corola de minuni a lumii/}[...]\textit{/căci eu iubesc/și flori și ochi și buze și morminte”}. Universul armonios este numit prin metafora revelatorie \textit{„corola de minuni”} și este o sumă permanentă de taine, imaginate ca petalele unei corole uriașe, care se revelează eului liric într-o enumerare metaforică: \textit{„flori”}, \textit{„ochi”}, \textit{„buze”}, \textit{„morminte”}.

\marginnote{relații de opoziție}[0.3cm]
Discursul liric se construiește în jurul relației de opoziție dintre cele două tipuri de cunoaștere, care se realizează prin antiteza \textit{eu/alții}, \textit{„lumina mea”}/\textit{„lumina altora”}, prin alternanța motivului luminii și al întunericului, evidențiate prin conjuncția \textit{„dar”}.

Motive poetice sunt: \textit{misterul și lumina}, motiv central în poezie și metaforă emblematică pentru opera poetică a lui \operaauthor, inclusă și în titlul volumului de debut, cu o dublă semnificație: \textit{cunoașterea} și \textit{iubirea}.

\marginnote{compoziție}[0.3cm]
Poezia cuprinde trei secvențe lirice (versurile 1--5; 6--8; 19--20). Acestea subliniază atitudinea poetului față de cunoaștere (\rom{1}), opoziția \textit{„lumina mea”} -- \textit{„lumina altora”} (\rom{2}) și explicația pentru atitudinea din prima secvență: \textit{„eu iubesc”} (\rom{3}).

Prima secvență (primele cinci versuri) exprimă concentrat refuzul cunoașterii logice, raționale, paradiziace, prin verbe la forma negativă: \textit{„nu strivesc”}, \textit{„nu ucid (cu mintea)”}. Enumerația de metafore revelatorii, cu multiple semnificații, desemnează temele poeziei lui \operaauthor: \textit{„flori”} -- viața/efemeritatea/frumosul, \textit{„ochi”} -- cunoașterea/spiritualitatea/contemplarea poetică a lumii, \textit{„buze”} -- iubirea/rostirea poetică, \textit{„morminte”} -- tema morții/eternitatea. Cele patru elemente pot fi grupate simbolic: \textit{„flori”} -- \textit{„morminte”} ca limite temporale ale ființei, \textit{„ochi”} -- \textit{„buze”} ca două modalități de cunoaștere: spirituală -- afectivă sau contemplare -- verbalizare.

A doua secvență, mai amplă, se construiește pe baza unor relații de opoziție: \textit{eu} -- \textit{alții}, \textit{„lumina mea”} -- \textit{„lumina mea”} -- \textit{„lumina altora”}, ca o antiteză între cele două tipuri de cunoaștere, paradiziacă și luciferică. Conjuncția adversativă \textit{„dar”}, reluarea pronumelui personal \textit{„eu”}, verbul la persoana \rom{1} singular, formă afirmativă, \textit{„sporesc”} afirmă opțiunea poetică pentru modelul cunoașterii luciferice.

Plasticizarea ideilor poetice se realizează cu ajutorul elementelor imaginarului poetic blagian: \textit{„lună”}, \textit{„noapte”}, \textit{„zare”}, \textit{„fiori”}, \textit{„mister”}. Câmpul semantic al misterului cuprinde cuvinte sau sintagme cu potențial revelator: \textit{„tainele”}, \textit{„nepătrunsul ascuns”}, \textit{„a luminii taină”}, \textit{„întunecata zare”}, \textit{„sfânt mister”}, \textit{„ne-nțeles”}, \textit{„ne-nțelesuri și mai mari”}.

Ultimele două versuri constituie o a treia secvență, cu rol conclusiv, deși exprimată prin raportul de cauzalitate (\textit{„căci”}). Cunoașterea poetică este un act de contemplație (\textit{„tot... se schimbă... sub ochii mei”}) și de iubire (\textit{„căci eu iubesc/și flori și buze și morminte”}).

\marginnote{înnoiri prozodice}[0.3cm]
Înnoirile prozodice moderniste sunt versul liber și ingambamentul. Poezia este alcătuită din 20 de versuri libere (cu metrică variabilă), al căror ritm interior redă fluxul ideilor poetice și frenezia sentimentelor.

\marginnote{limbaj poetic}[0.6cm]
Sursele expresivității și ale sugestiei se regăsesc la fiecare nivel al limbajului poetic. La nivel morfologic, repetarea cuvântului-cheie \textit{„eu”} susține definirea relației creator -- lume. Seriile verbale antitetice redau atitudinea față de mister ilustrate de cele două tipuri de cunoaștere: \textit{„lumina altora”} -- \textit{„sugrumă (vraja)”}, adică \textit{„strivește”}, \textit{„ucide”} (\textit{„nu sporește”}, \textit{„micșorează”}, \textit{„nu îmbogățește”}, \textit{„nu iubește”}); \textit{„lumina mea”} -- \textit{„sporesc”} (\textit{„a lumii taină”}), \textit{„mărește”}, \textit{„îmbogățesc”}, \textit{„iubesc”} (\textit{„nu sugrum”}, \textit{„nu strivesc”}, \textit{„nu ucid”}).

O trăsătură modernistă a limbajului poetic este intelectualizarea emoției. În poezia lui Blaga, limbajul și imaginile artistice sunt puse în relație cu un plan filosofic secundar.

\subsection{Concluzie}

În concluzie, arta poetică \operatitle\ de \operaauthor\ aparține modernismului printr-o serie de particularități de structură și de expresivitate: viziunea asupra lumii (subiectivismul), intelectualizarea emoției, influențele expresioniste, noutatea metaforei, tehnica poetică, înnoirile prozodice.
