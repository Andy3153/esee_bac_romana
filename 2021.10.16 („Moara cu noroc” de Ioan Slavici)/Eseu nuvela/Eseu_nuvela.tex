% Include the preamble
%%
%% Basic LaTeX preamble by Andy3153
%% created   04/10/21 ~ 18:21:32
%% modified1 15/10/21 ~ 23:15:05
%% modified2 03/11/21 ~ 18:37:45
%% modified3 22/02/22 ~ 12:48:17
%%
%% it used to be a template rip
%%
%% reguli de scriere:
%% 0. o recomandare mai mult, reține că am folosit xelatex, nu pdflatex
%%
%% 1. când apare [...] într-un citat, NU trebuie să fie italic și el
%%      exNU: \textit{„text1 [...] text2”}
%%      exDA: \textit{„text1} [...] \textit{text2”}
%%
%% 2. când apare o enumerație de citate, NU trebuie să fie italice și virgulele
%%      exNU: \textit{„citat1”, „citat2”, „citat3”}
%%      exDA: \textit{„citat1”}, \textit{„citat2”}, \textit{„citat3”}
%%
%% 3. să NU-ți fie frică să folosești \hbox dacă îti desparte latex aiurea cuvinte cu cratimă
%%
%% 4. să NU-ți fie frică să folosești \- ca să corectezi dacă îți desparte latex aiurea în silabe (destul de rar dacă folosești xelatex, de ce nu știu am luat-o ca atare, probabil ca are utf-8 by default și de aia)
%%
%% 5. în sectiunile care au un text în paranteză unde da exemple, formatează exemplul ăla astfel:
%%      ex: \section{text1 {\footnotesize\normalfont (de exemplu: text2, text3 etc.)}}
%%
%%
%%
%% 69.0. de șters despărțirile în silabe degeaba de când foloseam pdflatex
%%
%% 69.1. de convertit numerele romane în \rom{numar}
%%
%% 69.2. de convertit in preambul \operatitle in \textbfit
%%
%% 69.3. de convertit în text \opera{title,author} în \opera{title,author}\ unde e nevoie
%%
%% 69.4. de pus 2 enteruri la început se secțiune/subsecțiune nouă
%%
%% 69.5. de facut aia cu footnoteize in sectiuni
%%


\documentclass[
 12pt,                        % Font size
 a4paper                      % Paper type
]{article}


% Packages
\usepackage[
 margin=2.7cm,                % Margin size
 marginparwidth=2cm,          % Margin note size
 marginparsep=3mm             % Space between margin and text
]{geometry}
%\usepackage[utf8]{inputenc} % UTF-8 support, disabled because of switch to XeLaTeX
%\usepackage[T1]{fontenc}    % Proper hyphenation, disabled because of switch to XeLaTeX
\usepackage[romanian]{babel} % Romanian characters support
\usepackage{indentfirst}     % Add paragraph indentation even after a section
\usepackage{marginnote}      % Notes on the margins of a document (more advanced \marginpar)
\usepackage{titlesec}        % Customize titles
%\usepackage{ulem}            % More underlines
\usepackage[dash,dot]{dashundergaps}


% Custom format for titles, sections, subsections etc.
\titleformat*{\section}{\large\bfseries}
\titleformat{\subsection}{\normalfont\normalfont\bfseries}{}{1.5em}{}


% Page numbering
%\pagenumbering{gobble} % uncomment if you want to disable it


% Custom commands
% Format: \newcommand{\command}[variable]{action #variable}
\newcommand{\rom}[1]{\uppercase\expandafter{\romannumeral #1\relax}} % Roman numerals
\newcommand{\textbfit}[1]{\textbf{\textit{#1}}}                      % combine bold and italic
\newcommand{\ex}[1]{\dashuline{\texttt{\footnotesize#1}}}                              % things to be filled by the reader
\newcommand{\comment}[1]{}                                           % comments
\newcommand{\operatitle}{}                                           % to not get errors
\newcommand{\operaauthor}{}                                          % to not get errors
\newcommand{\ApplySubIIStyling}                                      % custom styling for a part of the document
{
 \setcounter{tocdepth}{1} % Now begin allowing sections to appear
 %\newcommand{\sectionbreak}{\ifnum\value{section}>0 \clearpage\fi}
}


% Customize \marginnote font
\renewcommand\marginfont{\ttfamily\footnotesize}


% Make \ttfamily hyphenate words for the margin notes
\DeclareFontFamily{OT1}{cmtt}{\hyphenchar\font=-1}
\DeclareFontFamily{\encodingdefault}{\ttdefault}{\hyphenchar\font=`\-}
\DeclareFontFamily{T1}{cmtt}{\hyphenchar\font=45}


% Make subsections not appear in ToC as we're using them for a completely different thing
\setcounter{tocdepth}{1}


% Basic document info
\date{}   % Show no date in the title
\author{} % Empty author to not get a warn about missing author


\newcommand{\operatitle}{\textbfit{„Moara cu noroc”}} % title of the text
\newcommand{\operaauthor}{Ioan Slavici} % author of the text

\title{Eseu cu privire la tema și viziunea despre lume dintr-o nuvelă studiată}


\begin{document}
\maketitle % Show the title
%\reversemarginpar % put margin notes on left instead of on right

% Beginning of text


\subsection{Context}

Publicată în 1881, în volumul de debut \textbfit{„Novele din popor”}, nuvela realistă, de factură psihologică \operatitle\ devine una dintre scrierile reprezentative pentru viziunea lui Ioan Slavici asupra lumii și asupra vieții satului transilvănean.


\section{Evidențierea trăsăturilor care fac posibilă încadrarea nuvelei într-o tipologie, într-un curent cultural/literar, într-o orientare tematică}

\operatitle\ de \operaauthor\ este o nuvelă, adică o specie epică în proză, cu un fir narativ central și o construcție epică riguroasă, cu un conflict concentrat. Personajele relativ puține scot în evidență evoluția personajului principal, puternic individualizat.

\marginnote{nuvelă psihologică}[0.3cm]
Este o nuvelă psihologică deoarece înfățișează frământările de conștiință ale personajului principal, care trăiește un conflict interior, moral și se transformă sufletește, iar analiza se realizează prin tehnici de investigare psihologică: monolog interior, stil indirect liber, scene dialogate, însoțite de notația gesticii și a mimicii.

\marginnote{nuvelă realistă}[0.8cm]
Este o nuvelă realistă prin: tema familiei și a dorinței de înavuțire, obiectivitatea perspectivei narative, includerea de personaje tipice pentru o categorie socială (Ghiță reprezintă tipul cârciumarului dornic de îmbogățire, Pintea este jandarmul, Lică este Sămădăul, dar și tâlharul), verosimilitatea, prezentarea veridică a societății ardelenești din a doua jumătate a secolului al \rom{19}-lea, tehnica detaliului semnificativ în descriere și în portretizare.


\section{Ilustrarea temei nuvelei studiate prin episoade/citate/sec\-ven\-țe comentate}

\operatitle\ de \operaauthor\ are ca temă consecințele nefaste și dezumanizante ale dorinței de îmbogățire. Din perspectiva socială, nuvela prezintă încercarea lui Ghiță de a-și schimba statutul social (din cizmar vrea să devină cârciumar) și de a asigura familiei sale un trai îndestulat. Din perspectivă moralizatoare, nuvela prezintă consecințele nefaste ale dorinței de a avea bani. Din perspectivă psihologică, nuvela prezintă conflictul interior trăit de Ghiță, care, dornic de prosperitate economică, își pierde treptat încrederea în sine și familie.


\section{Prezentarea elementelor de structură și de compoziție ale textului narativ, semnificative pentru tema și viziunea despre lume din nuvela studiată {\footnotesize\normalfont (de exemplu: acțiune, conflict, relații temporale și spațiale, incipit, final, tehnici narative, perspectivă narativă, registre stilistice, limbajul personajelor etc.)}}

Titlul nuvelei este mai degrabă ironic. Toposul ales, cârciuma numită Moara cu noroc, ajunge să însemne, mai curând, Moara cu ghinion, Moara care aduce nenorocirea, deoarece câștigurile obținute aici ascund nelegiuiri.

\marginnote{perspectivă narativă}[-0.2cm]
Perspectiva narativă este obiectivă. Întâmplările din nuvelă sunt relatate la persoana a \rom{3}-a, de către un narator atașat, omniscient și omniprezent.

Pe lângă perspectiva obiectivă a naratorului, apare tehnica punctului de vedere în intervențiile simetrice a bătrânei, din incipitul și finalul nuvelei. Soacra afirmă la început, într-o discuție cu Ghiță, că: \textit{„Omul să fie mulțumit cu sărăcia sa, căci, dacă-i vorba, nu bogăția, ci liniștea colibei tale te face fericit”}, iar la sfârșit pune întâmplările tragice din nuvelă pe seama destinului necruțător: \textit{„așa le-a fost data!..”}.

Cele două teze morale, formulate în prolog și epilog, sunt confirmate în desfășurarea narativă, iar personajele care încalcă aceste principii ale satului tradițional sunt sancționate în finalul nuvelei.

În dialogul din incipitul nuvelei, dintre soacră și Ghiță, se confruntă două concepții despre viață/fericire: bătrâna este adepta valorilor tradiționale, în timp ce Ghiță, capul familiei, dorește bunăstarea materială. Ghiță, cizmar sărac, dar cinstit și harnic, ia în arendă cârciuma de la Moara cu noroc, pentru a câștiga rapid bani, ca să-și deschidă un atelier. Cârciumarul nu este la început un om slab, ci dimpotrivă, voluntar, care își asumă responsabilitatea destinului celorlalți.

\marginnote{conflictul central}[0.3cm]
Fiind o nuvelă psihologică, în \operatitle\ de \operaauthor\ conflictul central este unul moral, psihologic, conflictul interior al protagonistului. Personajul principal, Ghiță, oscilează între dorința de a rămâne om cinstit, pe de o parte, și dorința de a se îmbogăți alături de Lică, pe de altă parte. Conflictul interior se reflectă în plan exterior, în confruntarea dintre cârciumarul Ghiță și Lică Sămădăul.

În nuvela realistă, spațiul și timpul sunt precizate. Cârciuma de la Moara cu noroc este așezată la răscruce de drumuri, izolată, înconjurată de pustietăți întunecoase. Acțiunea se desfășoară pe parcursul unui an, între două repere temporale cu valoare religioasă: de la Sfântul Gheorghe până la Paștele din anul următor; apa și focul purifică locul.

Alcătuită din 17 capitole, cu prolog și epilog, nuvela are un subiect concentrat.

\marginnote{expozițiunea}[0.3cm]
În expozițiune, descrierea drumului care merge la Moara cu noroc și a locului în care se află cârciuma fixează cadrul acțiunii. Ghiță, cizmar sărac, hotărăște să ia în arendă cârciuma de la Moara cu noroc, pentru a câștiga bani mai mulți și mai repede. O vreme, afacerile îi merg bine, iar primele semne ale bunăstării și ale armoniei în care trăiește familia nu întârzie să apară, scena numărării banilor, sâmbătă seara, fiind sugestivă.

\marginnote{intriga}[0.9cm]
Apariția lui Lică Sămădăul, șeful porcarilor și al turmelor de porci din împrejurimi, la Moara cu noroc, constituie intriga nuvelei, declanșând în sufletul lui Ghiță conflictul interior și tulburând echilibrul familiei. Lică îi cere să-i spună cine trece pe la cârciumă, iar Ghiță își dă seama că nu poate rămâne la Moara cu noroc fără acordul Sămădăului.

Mai întâi, Ghiță își ia toate măsurile de apărare împotriva lui Lică: merge la Arad să-și cumpere două pistoale, își face rost de doi câini și își angajează încă o slugă, pe Marți, \textit{„un ungur înalt ca un brad”}.

\marginnote{desfășurarea acțiunii}[1.2cm]
Desfășurarea acțiunii ilustrează procesul înstrăinării cârciumarului față de familie și al dezumanizării provocate de dorința de îmbogățire prin complicitatea cu Lică. Datorită generozității Sămădăului, starea materială a lui Ghiță devine tot mai înfloritoare, numai că Ghiță începe să-și piardă încrederea în sine. Cârciumarul devine interiorizat, mohorât, violent, îi plac jocurile crude, primejdioase, se poartă brutal față de Ana, pe care o protejase până atunci, și față de copii. La un moment dat, ajunge să regrete că are familie și copii, pentru că nu-și poate asuma total riscul îmbogățirii alături de Lică. Cârciumarul se aliază cu jandarmul Pintea, fost hoț de codru și tovarăș al lui Lică, pentru a-l da în vileag pe Sămădău, însă nu este cinstit nici față de acesta, căci dorește să își păstreze o parte din banii obținuți din afaceri necurate.

\marginnote{punctul culminant}[0.8cm]
Punctul culminant ilustrează dezumanizarea lui Ghiță. La sărbătorile Paștelui, Ghiță își aruncă soția în brațele Sămădăului, lăsând-o singură la cârciumă, în timp ce el merge să-l anunțe pe jandarm că Lică are asupra lui bani furați. Dezgustată de lașitatea soțului și neștiind motivul real pentru care acesta plecase, într-un gest de răzbunare disperată, Ana i se dăruiește lui Lică deoarece, spune ea, în ciuda nelegiuirilor comise, el e \textit{„om”}, pe când Ghiță \textit{„nu e decât muiere îmbrăcată în haine bărbătești”}.

\marginnote{deznodământul}[0.8cm]
Deznodământul este tragic. Dându-și seama că soția l-a înșelat, Ghiță o ucide pe Ana, fiind la rândul lui omorât de Răuț, din ordinul lui Lică. Un incendiu provocat de oamenii lui Lică mistuie cârciuma de la Moara cu noroc. Pentru a nu cădea viu în mâinile lui Pintea, Lică se sinucide izbindu-se cu capul de un copac. Nuvela are final moralizator, sancționarea protagoniștilor este pe măsura faptelor. Singurele personaje care supraviețuiesc sunt bătrâna și copiii, ființele morale și inocente.

În nuvelă, accentul nu cade pe actul povestirii, ci pe complexitatea personajelor, care par să aibă un destin prestabilit.

Personajul principal, Ghiță, este cel mai complex din nuvelistica lui Slavici, un personaj \textit{„rotund”}, care trăiește un proces al dezumanizării, cu frământări sufletești și ezitări. Destinul lui ilustrează consecințele nefaste ale dorinței de îmbogățire.

Lică rămâne pe parcursul nuvelei egal cu sine, \textit{„un om rău și primejdios”}. Sămădău și tâlhar, este necruțător cu trădătorii, generos cu aceia care îl sprijină în afacerile necurate, hotărât și crud.

Ana suferă esențiale transformări interioare care îi oferă scriitorului posibilitatea unei fine analize a psihologiei feminine. La început o femeie devotată căminului, protejată mai întâi de mamă și apoi de soț, reprezentând un ideal de feminitate, Ana este împinsă în brațele Sămădăului și apoi este ucisă de Ghiță, fiindcă l-a înșelat.

Trăsăturile personajelor se desprind din fapte, vorbe, gesturi și din relațiile care se stabilesc între acestea. De asemenea, naratorul realizează portrete sugestive, iar detaliile fizice relevă trăsături morale sau statutul social. Mijloacele de investigație psihologică sunt: scenele dialogate, monologul interior de factură tradițională și acela realizat în stil indirect liber, introspecția, notația gesticii, a mimicii și a tonului vocii.


\subsection{Concluzie}

\operatitle\ de \operaauthor\ este o nuvelă realistă și o nuvelă psihologică, pentru că urmărește conflictul interior, frământările în planul conștiinței personajelor.
\end{document}
